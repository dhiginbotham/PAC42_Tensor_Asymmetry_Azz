This proposal is an update to PR12-11-110 which was submitted to PAC38.  For convenience, we reproduce the PAC report on the next page.   We provide here an overview of the actions we've taken to address the PAC concerns. Full details are available in the main text.

As suggested by PAC38, we have modified our experimental technique to measure the tensor asymmetry instead of the cross section difference.  This takes the simplified form of the ratio of tensor polarized to unpolarized cross-sections shown in Eq.~\ref{3}.   While this cancels the largest first order effects\footnote{For example, the target magnetic field will be oriented along the beamline during both polarized and unpolarized data taking, which greatly reduces the sensitivity to changes in acceptance in the two configurations.}, special care will be needed to control the sensitivity of the integrated counts in each state to time dependent drifts in detector response, charge measurement and luminosity. 

We have assumed a tensor polarization (P$_{zz}$=\PZZ\%) which is larger than the previous proposal. This assumption is based on the documentation of tensor polarized targets previously discussed in publications, and is supported by the experience of the collaboration's polarized target groups.   This will require incremental development of existing DNP techniques.  We acknowledge that less established methods, such as the `hole-burning' technique recommended by the PAC, hold very good potential to produce significantly higher tensor polarization, but this will require significant R\&D.  We have initiated this process, although from a practical perspective, the funding for this development will likely remain limited until an approved experiment demonstrates the need for these novel tensor polarized targets. 

The $x_B$-coverage has been expanded, although we note that a significantly non-zero value of $b_1$ at any $x_B$ would unambiguously confirm its non-conventional behavior.  Finally, we have engaged several theorists for calculations and to confirm that our interpretation of the relationship between the measured asymmetry and the tensor structure function $b_1$ is valid.



\newpage
\subsection*{PAC38 Report}
{
\noindent
{\bf PR12-11-110} ``The Deuteron Tensor Structure Function b1''

\vspace{0.1cm}
\noindent
{\bf Motivation}: This proposal, a follow-up of LOI-11-003 submitted to PAC37, is dedicated to the measurement of the deuteron tensor structure function $b_1$ by measuring deep inelastic scattering from a tensor polarized deuterium target. All available models predict a small or vanishing value of $b_1$ at low x, however the first pioneering measurement of $b_1$ at HERMES revealed a crossover to an anomalously large negative value, albeit with a relatively large experimental uncertainty. This justifies the intention to make a precise measurement: confirmation that $b_1$ is relatively large may then require an explanation based on more exotic models for the deuteron, such as hidden color due to a 6-quark configuration.

%\vspace{1cm}
\noindent
{\bf Measurement and Feasibility}: The collaboration proposes to carry out this experiment in Hall C, using the polarized UVa/JLab ND$_3$ target, the HMS/SHMS spectrometers and an unpolarized 115 nA electron beam. The tensor structure function $b_1$ is derived from the measurement of the difference between the transversely and longitudinally tensor polarized cross-sections, which is directly proportional to $b_1$ itself. From the measured value of $b_1$ the tensor asymmetry $A_{zz}$ can be calculated, provided the structure function $F_1$ is known. The collaboration proposes to perform the measurement in 28 days of data taking at 11 GeV at the two x values of 0.3 and 0.5, which cover the range in which the HERMES data display the crossover of $b_1$ to large negative values.



%\vspace{1cm}
\noindent
{\bf Issues}: Despite the interesting physics case presented, the PAC has identified several issues with this proposal.
\begin{enumerate}
\item One obvious problem is the theoretical interpretation of the results of this kind of experiments. Following the recommendation of PAC37 the collaboration has partially addressed this question by expanding the discussion of the expected behavior of $b_1(x)$ in various theoretical models. However to draw significant conclusions from this measurement, also given the limited kinematical coverage (see below) chosen, would require further work.
\item The chosen x range, although overlapping with the region in which the HERMES results were obtained, does not seem sufficient to determine $b_1(x)$ in such a way as to unambiguously establish its conventional or exotic behavior. The PAC encourages the collaboration to explore the possibility to carry out the measurement using a large acceptance spectrometer covering a wider x range.
\item The PAC has concerns about the proposed experimental method using the cross section difference between the transversely and longitudinally tensor polarized target configurations. Given a 5-tesla field for this type of target, the effect on the acceptance due to the target field for these configurations can be quite different, and such systematic uncertainties due to the acceptance and other effects may well be larger than the effect that the proponents are trying to measure.
\item The proponents should pursue the tensor asymmetry measurement technique. Currently, the proposed target has a rather low tensor polarization ($\sim$10\%). It is crucial and important to pursue more vigorously techniques such as the RF ``hole’’ burning technique to improve the tensor polarization of the target.
%}
\end{enumerate}
}

