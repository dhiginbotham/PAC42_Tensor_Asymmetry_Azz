\subsection{Rendering Dilution Factor}
\label{dil}
To derive the dilution factor we first start with the ratio of 
polarized to unpolarized counts.
%equation used to obtain the observable in terms of each measured cross section.
%\begin{equation}
%\frac{A_{zz}P_{zz}}{2}=\left(\frac{\sigma^1-\sigma}{\sigma}\right).
%\end{equation}
In each case, the number of counts that are actually measured, and neglecting 
the small contributions of the thin aluminium cup window materials, NMR coils, etc.,
are
\begin{equation}
N_1=Q_1\varepsilon_1 {\cal A}_1 l_1[(\sigma_N+3\sigma_1)p_f+\sigma_{He}(1-p_f)],
\end{equation}
and
\begin{equation}
N=Q\varepsilon {\cal A}l[(\sigma_N+3\sigma)p_f+\sigma_{He}(1-p_f)].
\end{equation}
where $Q$ represents accumulated charge, $\varepsilon$ is the dectector 
efficiency, ${\cal A}$ the cup acceptance, and $l$ the cup length.  

For
this calculation we assume similar charge accumulation such that $Q\simeq Q_1$, 
and that the efficiencies stay constant, in which case all factors drop out of 
the ratio leading to
\begin{eqnarray}
\nonumber \frac{N_1}{N}& = &\frac{{(\sigma_N+3\sigma_1)p_f+\sigma_{He}(1-p_f)}
}{(\sigma_N+3\sigma)p_f+\sigma_{He}(1-p_f)}\\
\nonumber & = & \frac{{(\sigma_N+3\sigma(1+2A_{zz}P_{zz}/2))p_f+\sigma_{He}(1-p_
f)}}{(\sigma_N+3\sigma)p_f+\sigma_{He}(1-p_f)}\\
\nonumber & = & \frac{{[(\sigma_N+3\sigma)p_f+\sigma_{He}(1-p_
f)]+3\sigma A_{zz}P_{zz}/2}}{(\sigma_N+3\sigma)p_f+\sigma_{He}(1-p_f)}\\
\nonumber & = & 1 + \frac{3\sigma 
A_{zz}P_{zz}/2}{(\sigma_N+3\sigma)p_f+\sigma_{He}(1-p_f)}\\
& = & 1 + \frac{1}{2} f A_{zz}P_{zz}, 
\end{eqnarray}
where $\sigma_1 = \sigma(1+2A_{zz}P_{zz}/2)$ has ben substituted, per 
eq.~(\ref{eq:one}), with $P_B =0$. It can be seen that the above result 
corresponds to eq.~(\ref{3}) in the main text.

