%Following the approach of Ref.~\cite{Frankfurt:1983qs},
M. Sargsian~\cite{MISAK} recently calculated the tensor asymmetry $A_{zz}$ for deep inelastic scattering.   
%Within the  PWIA approximation in which parton distributions in the deuteron  are generated
%due to partons in proton and neutron, he predicts no significant tensor asymmetry except at
%very large x (quasi-elastic scattering) in which case the D-wave becomes important. See Fig.~\ref{PROJ}.
%
See Fig.~\ref{PROJ}.  In the approximation in which  only proton-neutron component of the deuteron is taken into account and  nuclear parton distributions
are generated through the convolution of partonic distribution of nucleon and deuteron density matrix (see e.g. Refs.~\cite{Frankfurt:1981mk,Sargsian:2001gu}), %       FS81,SSS},   
the deuteron structure function $b_1$ is related directly to the d-partial wave of the deuteron wave function~\cite{MISAK,Frankfurt:1981mk}.   %\cite{FS81,MS}.  
As a result,  this approximation predicts negligible  magnitude for $b_1$  for $x\le 0.6 $ due to small Fermi momenta involved  in the convolution integral. 
However, the predicted magnitude of $b_1$ is large at $x \ge 0.7$ where one expects substantial contribution from the d-waves due to 
high momentum component of the deuteron wave function involved in the convolution picture of DIS scattering off the deuteron.
In this case, $b_1$ is very sensitive to the relativistic description of the deuteron and its measurement can be used for checking 
the different approximations of high momentum  component of deuteron wave function.  

In the calculation presented, two Virtual Nucleon  and Light-Cone approximations are used to calculate the  tensor polarization for 
DIS scattering off the deuteron.  In both approximations only the proton-neutron component of the deuteron is taken into account.
In the Virtual Nucleon approximation, the covariant scattering amplitude is reduced  by estimating the spectator nucleon 
propagator at its on-energy shell in the lab frame of the deuteron.  Within this approximation the baryonic sum rule is satisfied while the 
momentum sum rule is not. The latter is due to the fact that part of the light cone momentum of the bound  virtual nucleon is lost to the 
unaccounted non-nucleonic degrees of freedom in the deuteron wave function.
In the light cone approximation the scattering amplitude is estimated the $E+p_z$ pole of the spectator nucleon on the light cone.
In this case the wave function is defined on the light-cone reference frame and it satisfies both baryon number and momentum sum rules.
For the detailed comparison  of these approximations, see Ref.~\cite{Sargsian:2001gu}.   %~\cite{SSS}.

