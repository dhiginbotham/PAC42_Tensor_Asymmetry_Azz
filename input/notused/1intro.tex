{\it UNDER CONSTRUCTION}

In this case, the familiar structure functions  $F_1$, $F_2$, $g_1$, and $g_2$ which describe 
inclusive scattering of electrons from spin-1/2 targets, must be supplemented for a spin-1 system 
by four additional structure functions : $b_1$, $b_2$, $b_3$, and $\Delta$.

The deuteron is the simplest many-body nuclear system.

The tensor structure function $b_1$, ( which is leading-twist like  $F_1$ and $g_1$), is quite 
interesting, in that it presents a simple gauge of nuclear effects: $b_1$ would vanish if the 
deuteron was simply a proton and neutron in a relative S state.

Nuclear effects/EMC effect
%When spin physics joins naturally the nuclear effects area and model independent 
%nuclear effects extraction compared to polarized EMC effect. 

Exotic components.

The Hermes collaboration  made a first measurement~\cite{Airapetian:2005cb} of 
$b_1$ and found significantly non-zero results.
Beyond providing insight into nuclear structure, this has the potential to impact $g_1^n$ and 
$g_2^n$ extractions, where $b_1$ has traditionally been ignored when the neutron is extracted 
from deuteron data.
