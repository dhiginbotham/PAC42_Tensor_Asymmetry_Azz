\subsection{Tensor Asymmetry Azz}


The S- and D-states are related to the tensor asymmetry $A_{zz}$ by~\cite{Frankfurt:1988nt}
\begin{equation}
	A_{zz} \propto \frac{\frac{1}{2}w^2(k)-u(k)w(k)\sqrt{2}}{u^2(k)+w^2(k)},
\end{equation}
where $u(k)$ is the S-state wave function and $w(k)$ is the D-state wave function.

For decades~\cite{PhysRev.81.165}, it has been known that the nucleon-nucleon potential has a short-range repulsive core, which is responsible for the stability of strongly interacting matter. However, a description of the repulsive core remains largely unconstrained and our understanding of QCD dynamics at short distances ($\leq 0.5\mathrm{~fm}$) largely incomplete~\cite{Sargsian:2014bwa}. 

Due to their small size~\cite{needed} and simple structure, tensor polarized deuterons are ideal for studying nucleon-nucleon interactions. Tensor polarization enhances the D-state wavefunction, which compresses the deuteron from $\sim?\mathrm{~fm}$ to $\sim0.5\mathrm{~fm}$~\cite{Forest:1996kp} and has been noted to be revealing of short-range QCD effects.
