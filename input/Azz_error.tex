%\subsubsection{Statistical error calculations of $A_{zz}$ and $b_1^d$}
\label{APPERR}
Full details of the error calculation can be found in Ref.~\cite{SOLVI}.

From section 6 of Ref.~\cite{Hoodbhoy:1988am},
%Hoodbhoy, Jaffe and Manihar (Nuc. Phys. B312, p571-588, 1989), 
we have:


\begin{eqnarray}
\frac{d\sigma_{\parallel}^H}{dxdy} & = & K \Bigg[x F_1(x) + \Big(\frac{2}{3} - H^2\Big) x b_1(x) \Bigg]\\
\frac{d\sigma_{\perp}^H}{dxdy}    & = & K \Bigg[x F_1(x) - \Big(\frac{1}{3} - \frac{1}{2}H^2\Big) x b_1(x) \Bigg]
\label{xs} 
\end{eqnarray}
%
with $K = \frac{e^4 M E}{2 \pi Q^4} [1+(1-y)^2]$.
%
For simplicity, we use $\sigma_{\parallel}$ for $\frac{d\sigma_{\parallel}^H}{dxdy}$ and $\sigma_{\perp}$ for $\frac{d\sigma_{\perp}^H}{dxdy}$.

%From Jaffe's email, 
We know that $H^2 = (P_z+2)/3$, where $P_z$ is the vector polarization of the target. 
%This is where is my doubt. From the paper and Jaffe's email, they define $H$ as the polarization along the beam. So my understanding is that $P = P_z$ and it is the vector polarization. 
%
And the tensor polarization $P_{zz}$ is related to $P_z$ via Eq.~\ref{TENSORVECTOR}.
%\begin{eqnarray}
%P_{zz} = 2 - \sqrt{4 - 3 P_z^2}
%\label{none} 
%\end{eqnarray}

The tensor asymmetry $A_{zz}$ depends on $b_1$ and $F_1$:
\begin{eqnarray}
\frac{b_1}{F_1} = - \frac{3}{2} A_{zz}
\label{none} 
\end{eqnarray}


Working with the equations of $\sigma_{\parallel}$ and $\sigma_{\perp}$, we can isolate $A_{zz}$ to find: 
\begin{eqnarray}
\frac{\sigma_{\parallel} - \sigma_{\perp}}{\sigma_{\parallel} + 2 \sigma_{\perp}} = \frac{1}{4} P_z A_{zz}
\label{MAIN} 
\end{eqnarray}

Note, that if the vector polarization in the parallel orientation $P_z^\parallel$
differs from the polarization in the perpendicular orientation
$P_z^\perp$, then Eq.~\ref{MAIN} is modified slightly to:
\begin{eqnarray*}
\frac{\sigma_{\parallel} - \sigma_{\perp}}{\kappa\sigma_{\parallel} + 2 \sigma_{\perp}} = \frac{1}{4} P_z^\parallel A_{zz}
\end{eqnarray*}
where $\kappa={P_z^\perp}/{P_z^\parallel}$.   We've assumed $\kappa=1$ for rates calculations.



In order to calculate the statistical error on $A_{zz}$ we start from:
\begin{eqnarray}
(\delta A_{zz})^2 = \Bigg( \frac{\delta A_{zz}}{\delta \sigma_{\parallel}} \Bigg)^2 (\delta \sigma_{\parallel})^2 + \Bigg( \frac{\delta A_{zz}}{\delta \sigma_{\perp}} \Bigg)^2 (\delta \sigma_{\perp})^2
\label{none} 
\end{eqnarray}
and
\begin{eqnarray*}
\frac{\delta A_{zz}}{\delta \sigma_{\parallel}} & = &\frac{4}{P_z} \Bigg[\frac{- (\sigma_{\parallel} - \sigma_{\perp})}{(\sigma_{\parallel} + 2 \sigma_{\perp})^2} + \frac{1}{\sigma_{\parallel} + 2 \sigma_{\perp}} \Bigg] \\
         & = & \frac{4}{P_z} \frac{3 \sigma_{\perp}}{(\sigma_{\parallel} + 2 \sigma_{\perp})^2}
\label{none} 
\end{eqnarray*}


\begin{eqnarray*}
\frac{\delta A_{zz}}{\delta \sigma_{\perp}} & = &\frac{4}{P_z} \Bigg[\frac{- 2 (\sigma_{\parallel} - \sigma_{\perp})}{(\sigma_{\parallel} + 2 \sigma_{\perp})^2} - \frac{1}{\sigma_{\parallel} + 2 \sigma_{\perp}} \Bigg] \\
         & = & \frac{4}{P_z} \frac{-3 \sigma_{\parallel}}{(\sigma_{\parallel} + 2 \sigma_{\perp})^2}
\label{none} 
\end{eqnarray*}
%
to arrive at:
\begin{eqnarray}
(\delta A_{zz})^2 = \Bigg(\frac{4}{P_z}\Bigg)^2 \Bigg[ \frac{9 \sigma_{\perp}^2 \delta \sigma_{\parallel}^2 + 9\sigma_{\parallel}^2 \delta \sigma_{\perp}^2 }{(\sigma_{\parallel} + 2 \sigma_{\perp})^4} \Bigg]
\label{none} 
\end{eqnarray}

The parallel and perpendicular cross sections have the same kinematical weight $K$. Since $b_1$ is very small compared to $F_1$ (or equivalently, $A_{zz}$ is very small), we can make the assumption $\sigma_{\parallel} \approx \sigma_{\perp} \equiv \sigma$ and  $\delta \sigma_{\parallel} \approx  \delta \sigma_{\perp} \equiv \delta \sigma$. 

\begin{eqnarray}
(\delta A_{zz})^2 = \frac{9 \times 16}{P_z^2} \frac{2 \sigma^2 \delta \sigma^2}{(3 \sigma)^4} = \frac{32}{9 P_z^2} \frac{\delta \sigma^2}{\sigma^2}
\label{none} 
\end{eqnarray}

\begin{eqnarray}
\delta A_{zz} = \frac{4 \sqrt{2}}{3 P_z} \frac{\delta \sigma}{\sigma} =  \frac{4 \sqrt{2}}{3 P_z} \frac{1}{\sqrt{N}}
\label{none} 
\end{eqnarray}

We determine $N$ from the unpolarized cross section model~\cite{Martin:2009iq}, which then allows us to calculate the rates and the time.

\begin{eqnarray}
N = \frac{32}{9} \frac{1}{P_z^2 (\delta A_{zz}^{meas})^2}
\label{none} 
\end{eqnarray}

To get the rates as a function of the theoretical tensor asymmetry, we need to apply the dilution factors:
\begin{eqnarray}
A_{zz}^{meas} = f P_{zz} A_{zz}
\label{none} 
\end{eqnarray}

\begin{eqnarray}
N = \frac{32}{9} \frac{1}{P_z^2 (f P_{zz} \delta A_{zz})^2}
\label{none} 
\end{eqnarray}

We need $N/2$ events in parallel and perpendicular kinematics. If we had a pure deuterium target, the time need will be:
\begin{eqnarray}
T = \frac{N}{R_D} = \frac{32}{9} \frac{1}{R_D P_z^2 (f P_{zz} \delta A_{zz})^2}
\label{none} 
\end{eqnarray}

Now the deuterium rates are estimated from the unpolarized deuteron cross section model~\cite{Martin:2009iq} $\sigma_D$:
\begin{eqnarray}
 R_D = \sigma_D~dp~d\Omega~L = \sigma_D~dp~d\Omega~\rho_D~\frac{I}{e}
\label{none} 
\end{eqnarray}
%
with $\rho_D = \rho_{LiD} \cdot f_{LiD} \cdot PF_{LiD}$, where $f_{LiD}$ is the dilution and $PF_{LiD}$ is the packing fraction.




