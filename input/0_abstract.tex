

The tensor-polarized target asymmetry, $A_{zz}$, which is used to extract $b_1$ in the DIS region through the D($e,e'$)X channel, can be used to extract information on nucleon-nucleon interactions in the quasi-elastic region. The reaction is unique in that it can probe color transparency, which has never been explored at Jefferson Lab, and improve understanding of the deuteron wave function and particularly probe how short range correlations arise from proton-neutron interactions.

In the quasi-elastic region, $A_{zz}$ was first calculated in 1988 by Frankfurt and Strikman, using the Hamada-Johnstone and Reid soft-core wave functions \cite{Frankfurt:1988nt}. Recent calculations by {M.~Sargsian} revisit $A_{zz}$ in the $x>1$ range using virtual-nucleon and light-cone methods, which differ by up to a factor of two \cite{MISAK}.


An experimental determination of $A_{zz}$ could be performed utilizing the set-up available for E13-12-011 at five different $Q^2$ values over the course of 24 days, with [NUMBER] days of commissioning. The measurements are less sensitive to the target polarization than E13-12-011, such that this experiment could be used to prove that the condition of 30\% in-beam polarization is met for E13-12-011.