``The most direct evidence for tensor correlations in nuclei comes from measurements of the deuteron structure functions and tensor polarization by elastic electron scattering~\cite{Gilman:2001yh}. In essence, these measurements have mapped out the Fourier transforms of the charge densitites of the deuteron in states with spin projections $\pm1$ and 0, showing that they are very different."~-R. Schiavilla, et al.~\cite{Schiavilla:2006xx}


``The cross section for the double scattering process can be written as~\cite{Arnold:1979cg}
\begin{dmath}
	\frac{d\sigma}{d\Omega d\Omega_2} = \left. \frac{d\sigma}{d\Omega d\Omega_2}\right|_0 \left[1 + \frac{3}{2}hp_xA_y\sin{\phi_2} + \frac{1}{\sqrt{2}}t_{20}A_{zz} - \frac{2}{\sqrt{3}}t_{21}A_{xz}\cos{\phi_2}+\frac{1}{\sqrt{3}}t_{22}\left( A_{xx} - A_{yy} \right) \cos{2\phi_2}  \right]
\end{dmath}
where $h=\pm 1/2$ is the polarization of the incoming electron beam, $\phi_2$ the angle between the two sattering planes (defined in the same way as the $\phi$ shown in figure 24) and $A_y$ and the $A_{ij}$ are the vector and tensor analysing powers of the second scattering. Although there is a $p_z$ component to the vector polarization, the term is omitted from equation (25) as there is no longitudinal vector analysing power; without spin precession, this term cannot be determined."~-R. Gilman and F. Gross~\cite{Gilman:2001yh}

``Accurate [form factor] measurements require that $Q^2$ be known accurately since $A$ and $B$ vary rapidly with $Q^2$. Energy or angle offsets of a few times $10^{-3}$ could lead to $Q^2$ being off by up to $0.5\%$. For both $A$ and $B$, this leads to offsets that increase with $Q^2$, reaching about 2\% at $Q^2=1\mathrm{~GeV}^2$ and 4\% at $Q^2=6\mathrm{~GeV}^2$."~-R. Gilman and F. Gross~\cite{Gilman:2001yh}

``The body of [$A$] data, aside from the lowest $Q$ Orsay point, suggests the correctness of the Saclay measurements. Theoretical predictions span the range between the two data sets, and do not help to determine which is correct. Thus, a new high-precision experiment in this [higher] $Q^2$ range appears desirable."~-R. Gilman and F. Gross~\cite{Gilman:2001yh}