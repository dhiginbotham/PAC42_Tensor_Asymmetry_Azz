\subsection{Experimental Method}
\setcounter{footnote}{2}
Following Ref.~\cite{Hoodbhoy:1988am},  we will extract $b_1(x)$ from
%the asymmetry $A_{zz}$ by taking 
the difference of parallel and perpendicular polarized target cross sections, 
with an unpolarized\footnote{Polarized beam is not required for this experiment, but would enable a simultaneous measurement of $g_1$.  The
tensor structure function $b_1$ can then be isolated by averaging the results of data with beam polarized parallel and anti-parallel.  Any contribution from residual vector polarization can be eliminated by grouping together two sets of data of opposing beam helicity.  False asymmetries will be monitored in a similar fashion by periodically flipping the target spin direction.} incident electron beam.  
The polarized cross sections\footnote{For simplicity, we will use $\sigma_{\parallel}$ for $\frac{d\sigma_{\parallel}^H}{dxdy}$ and $\sigma_{\perp}$ for $\frac{d\sigma_{\perp}^H}{dxdy}$.} 
%$d^2\sigma_{\parallel}/d\Omega dE'$ and $d^2\sigma_{\perp}/d\Omega dE'$
can be extracted from data collected by scattering an unpolarized electron beam off a spin-1 target
polarized longitudinally and perpendicularly to the electron beam direction.
%

For this configuration, it can be shown that:
%\begin{eqnarray}
%\frac{\frac{d^2\sigma_{\parallel}}{d\Omega dE'} - \frac{d^2\sigma_{\perp}}{d\Omega dE'}}
%{\frac{d^2\sigma_{\parallel}}{d\Omega dE'} + 2 \frac{d^2\sigma_{\perp}}{d\Omega dE'}} = 
%- \frac{1}{2} (1 - \frac{3}{2} H^2) A_{zz},
%\label{para-perp}
%\end{eqnarray}
%
%where $H^2=(P+2)/3$, with $P$ the deuteron vector polarization.
\begin{eqnarray}
 \sigma_{\perp} - \sigma_{\parallel} = \frac{K}{6} (2 P_z^{\parallel} + P_z^{\perp}) x b_1
\label{MAIN}
\end{eqnarray}
where $P_z^{\parallel}$ and $P_z^{\perp}$ are the vector polarization achieved in the longitudinal and transverse configurations respectively. 
%
The tensor asymmetry can then be extracted from the structure function $b_1$ via:
\begin{eqnarray}
\label{STUFF}
A_{zz} = -\frac{2}{3} \frac{b_1}{F_1} 
\label{Azz}
\end{eqnarray}
The time necessary to achieve the desired precision $\delta b_1$ is:
\begin{eqnarray}
T = \frac{N}{R_D} = \frac{2}{\epsilon^2 R_D (\delta b_1 / b_1)^2}
\label{none}
\end{eqnarray}
%
where $R_D$ is the deuteron rate, which is estimated from Ref.~\cite{Martin:2009iq}, and $\epsilon$ is given  by Eq.~\ref{EPSILON}. 
Full details of the statistical error calculation are provided in Appendix~\ref{APPERR}, for reference.


%From Eq.~\ref{para-perp}, the beam time needed to achieve an absolute uncertainty 
%of $\delta A_{zz}$ can be deduced as:
%\begin{eqnarray}
%T = \frac{32}{9 P_{z}^2} \frac{1}{R_D~f~(P_{zz}~\delta A_{zz})^2}
%\label{time-eq2}
%\end{eqnarray}
%
%where $R_D$ is the deuteron rate, $f$ the dilution factor and $P_{zz}$ the tensor polarization.



\subsubsection{Systematic Uncertainties}
Table~\ref{sys} summarizes the systematic error estimate for the cross section measurement.
The unpolarized structure function
$F_1$ is used to evaluate $A_{zz}$ from the measured $b_1$  using Eq.~\ref{STUFF}.   This contributes an additional 5\% relative, which raises the total systematic for $A_{zz}$ to 10.6\%.

%\begin{table}
%\begin{center}
%\begin{tabular}{l|c}\hline\hline
%Item                         & Systematic \\
%\hline
%Polarimetry                  &   8\%   \\
%Dilution/packing fraction    &   5\%   \\
%Radiative corrections        &   5\%   \\
%%F$_1$ structure function     &   5\%   \\
%Computer deadtime            &  0.5\%  \\
%Charge measurement           &  0.5\%  \\
%Energy measurement           &  0.05\% \\
%\hline
%Total  &  10.7\%   \\
%\hline
%\end{tabular}
%\caption{\label{sys}Relative systematic uncertainties for $A_{zz}$.  }
%\end{center}
%\end{table}

\begin{table}
\begin{center}
  \begin{tabular}{ll} \hline\hline
 Source    &  (\%) \\
  \hline \hline
   Target Polarization                  & 5.0     \\
   Dilution/Packing fraction            & 5.0      \\
   Radiative Corrections                & 4.0     \\
   Acceptance                           & 4.0      \\
   Charge determination      & $\le$1      \\
   VDC efficiency            & $\le$1      \\
   PID detector efficiencies & $\le$1 \\
   Software cut efficiency   & $\le$1 \\
   Energy                    & 0.5      \\
 \hline\hline
    Total                    & 9.3  \\
 %\hline
 \end{tabular}
\caption{\label{sys} Major contributions to the cross section systematic.}
 \end{center}
\end{table}



%\begin{table}
%\begin{center}
%\begin{tabular}{c|ccc}\hline\hline
%     ~                 &  Time   & Number &   Total Time \\
%     ~                 &  (hrs)  &   ~    &    (hrs)     \\
%\hline
%Elastic calibration                &    48.0    &   1    &     48.0     \\
%Target dilution measurement               &    8.0    &   1    &     8.0      \\
%Configuration changes   &   16    &   4    &     64.0     \\
%Beam Energy measurement      &   2.0   &   2    &     4.0      \\
%BCM calibration        &   1.0   &   2    &     2.0      \\
%Target Annealing       &   2.5   &   4    &     10.0     \\
%Target Material Change &   4.0   &   4    &     16.0     \\
%\hline\hline
%\end{tabular}
%\caption{\label{OVHEAD}Summary of the Overhead.}
%\end{center}
%\end{table}


\begin{table}
\begin{center}
  \begin{tabular}{lrrr} \hline\hline
 Overhead & Number&Time Per (hr)&(hr)\\
\hline
Target anneal             &   30&       2.0&      60.0\\
Target field rotation     &    3&       12.0&      36.0\\
Beamline survey           &    2&       8.0&      16.0\\
Target material change    &    5&       8.0&      40.0\\
Target T.E.               &   16&       4.0&      64.0\\
Packing Fraction          &    6&       2.0&     12.0\\
\hline
%Pass change              &    0&       4.0&       0.0\\
Linac change              &    2&       8.0&      16.0\\
Momentum/angle change     &    1&       2.0&       2.0\\
Moller measurement        &    6&       2.0&       12.0\\
Optics                    &    3&       4.0&      12.0\\
Arc Energy Meas.          &    3&       2.0&       6.0\\
BCM calibration           &    2&       2.0&      8.0\\
\hline
                          &     &          &        \overhead_days days  \\
\hline
 \end{tabular}

 \end{center}

  \caption{\label{OVERHEAD} Major contributions to the overhead.}
\end{table}


%\subsection{Alternate Methodology}
%
%In addition to the experimental approach considered in this proposal, we 
% have two other options which we will explore further in preparation for a full proposal: 
%\begin{enumerate}
% \item measuring the tensor structure function $b_1$ with a longitudinally polarized target
%using the cross section method suggested in Ref.~\cite{Hoodbhoy:1988am};
% \item measuring the tensor asymmetry $A_{zz}$ using the HERMES method~\cite{Riedl:2005jq}, 
%with only a longitudinally polarized target.
%\end{enumerate}
%We are also exploring other forms of accessing $A_{zz}$ that don't rely on $A_2^d$ 
%being negligible, like HERMES assumed.




\subsubsection{Overhead}
%
Table~\ref{OVERHEAD} summarizes the expected overhead, which sums to \overhead_days days.
%In order to calibrate the target polarimetry, elastic scattering measurements will be performed at %an 
%incident energy of 2.2 GeV. 
Measurements
of the dilution from the unpolarized materials contained in the target, and of the packing fraction due to
the granular composition of the target material will be performed with a carbon target.
Target annealing will be performed approximately once per day, and target material changes will be performed slightly more than  once a week.
Configuration changes include rotation of the magnetic field of the target from parallel to perpendicular and vice versa.


%\subsection{Background}
%The pion background has not been estimated yet for this measurement, but should be comparable to other
%proposed DIS measurements with HMS and SHMS. A careful study of the background will be performed in a full
%proposal.
