\documentclass[
%twocolumn,
%preprintnumbers,
preprint,
aps,prd,
nofootinbib,
superscriptaddress,
tightenlines,
amsmath,
amssymb
]{revtex4}
\usepackage{bm}
\usepackage[dvips]{color,graphicx}
\usepackage{amsmath}
\usepackage{amssymb}

\newcommand{\ksqmax}{k^2_{\mathrm{max}}}

%
\begin{document}


%\maketitle
\baselineskip 3.0ex
%\begin{abstract}%
A measurement of $b_1$ is further motivated by its connection with the spin 1 angular momentum sum rule [1]. 
First of all, by examining the energy momentum tensor for the deuteron in Ref.[1] it was possible to define an additional sum rule for $b_1$ (see Eq.(12) in Ref.[1]) where it was shown that the second moment of this quantity is non vanishing, being related to one of the gravitomagnetic deuteron form factors.  This experiment would provide a unique test of this idea.
It is also important to notice that $b_1$ singles out the role of the $D$-wave component in distinguishing coherent nuclear effects through tensor polarized correlations  from  the independent nucleon's partonic spin structure. 
A similar role of the D-wave component was also found in the recently proposed spin sum rule where it plays a non trivial role 
producing a most striking effect through the spin flip GPD E. An experimental measurement of $b_1$ would corroborate this scenario.  


[1]  S.~K.~Taneja, K.~Kathuria, S.~Liuti and G.~R.~Goldstein, \\
{\em  ``Angular momentum sum rule for spin one hadronic systems''} \\ 
Phys.\ Rev.\ D {\bf 86}, 036008 (2012);
 arXiv:1101.0581 [hep-ph]

\end{document}
