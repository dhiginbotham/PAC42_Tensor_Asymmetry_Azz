\subsection{Overhead}

Table~\ref{OVERHEAD} summarizes the expected overhead, which sums to \overheaddays days.
%In order to calibrate the target polarimetry, elastic scattering measurements will be performed at %an
%incident energy of 2.2 GeV.
The dominant overhead comes from switching from the polarized to unpolarized state and vice versa, and target anneals.  The target will need to be annealed about every other day, and the material replaced once a week.
Measurements of the dilution from the unpolarized materials contained in the target, and of the packing fraction due to the granular composition of the target material will be performed with a carbon target.

%Configuration changes include rotation of the magnetic field of the target from parallel to perpendicular and vice versa.

\begin{table}
\begin{center}
  \begin{tabular}{lrrr} \hline\hline
 Overhead & Number&Time Per (hr)&(hr)\\
\hline
Polarization/depolarization & 30&       2.0&     60.0\\
Target anneal             &   13&       4.0&     52.0\\
Target T.E. measurement   &    5&       4.0&     20.0\\
%Beamline survey          &    2&       8.0&     16.0\\
Target material change    &    4&       4.0&     16.0\\
Packing Fraction/Dilution runs &    6&       1.0&      6.0\\
\hline
%Pass change              &    0&       4.0&       0.0\\
BCM calibration           &    8&       2.0&      16.0\\
Optics                    &    3&       4.0&      12.0\\
Linac change              &    1&       8.0&      8.0\\
Momentum/angle change     &    3&       2.0&       6.0\\
%Arc Energy Meas.          &    3&       2.0&       6.0\\
\hline
                          &     &          &        \overheaddays days  \\
\hline
 \end{tabular}
 \end{center}
  \caption{\label{OVERHEAD} Major contributions to the overhead.}
\end{table}
