


\subsection{Deuteron Form Factors}

``Accurate [form factor] measurements require that $Q^2$ be known accurately since $A$ and $B$ vary rapidly with $Q^2$. Energy or angle offsets of a few times $10^{-3}$ could lead to $Q^2$ being off by up to $0.5\%$. For both $A$ and $B$, this leads to offsets that increase with $Q^2$, reaching about 2\% at $Q^2=1\mathrm{~GeV}^2$ and 4\% at $Q^2=6\mathrm{~GeV}^2$."~-R. Gilman and F. Gross~\cite{Gilman:2001yh}

``The body of [$A$] data, aside from the lowest $Q$ Orsay point, suggests the correctness of the Saclay measurements. Theoretical predictions span the range between the two data sets, and do not help to determine which is correct. Thus, a new high-precision experiment in this [higher] $Q^2$ range appears desirable."~-R. Gilman and F. Gross~\cite{Gilman:2001yh}

``We are forced to conclude that these high $Q^2$ [form factor] measurements \emph{cannot be explained by nonrelativistic physics and present very strong evidence for the presence of interaction currents, relativistic effects or possibly new physics}."~-R. Gilman and F. Gross~\cite{Gilman:2001yh}

``It is clear that much more work will be needed to clarify the various physics issues, before a convergent scheme is established for treating the e.m. and strong interaction physics properly."~-J.A. Tjon~\cite{Tjon.48218}

``It turns out that this leading twist pQCD estimate is $10^3-10^4$ times smaller than the measured deuteron form factor, implying large soft contributions to the form factor, in agreement with \cite{Isgur:1988iw,Radyushkin:1990te}, suggesting that pQCD should not be used as an explanation for the form factor. The calculation is extremely complicated and a confirmation, or refutation, is desirable."~-R. Gilman and F. Gross~\cite{Gilman:2001yh}

``From the discussions in section 3.8, it is clearly of interest to extend measurements of $A$ to higher $Q^2$. An $ed$ coincidence experiment is straightforward, but prohibitive timewise with present accelerators. The proposed 12~GeV JLab upgrade allows one to take advantage of the approximate $E^2$ scaling of $\sigma_M$ at constant $Q^2$ and high energy~\cite{Petratos:2002wz}. A large acceptance spectrometer such as MAD would be very helpful. Depending on the details of the upgrade, a one month experiment could provide data to $Q^2$ of 8 GeV$^2$."~-R. Gilman and F. Gross~\cite{Gilman:2001yh}


\subsection{Light Cone Calculations}

``It is now known that the tensor part of the one-pion exchange interaction is too strong to be treated pertubatively, and recent work has focused on how to include the singular parts of one-pion exchange in the most effective manner~\cite{Phillips:1999hh,Phillips:1999am,Walzl:2001vb}"~-R. Gilman and F. Gross~\cite{Gilman:2001yh}

``But a principal motivation for using the front-form is that it is a natural choice at very high momentum, where the interactions single out a preferred direction (the beam direction) and the dynamics evolves along the light-front in that direction. The disadvantage is that the generators that contain dynamical quantities ar $H_{-}$ and $J^i$, and this means that angular momentum conservation must be treated as a dynamical constraint."~-R. Gilman and F. Gross~\cite{Gilman:2001yh}

``One of the fundamental problems in field theory is the description of the quantum bound states in a relativistically invariant form. The problem inevitably arises with the proper treatment of the time at which the bound state is observed as well as the unambiguous identification of the compositeness of a bound state due to the non-trivial structure of the vacuum in field theories. The latter is related to the differentiation of the constituents of the bound system from the particles arising from the vacuum fluctuations."~-M. Sargsian~\cite{Sargsian:2014bwa}




\subsection{Guides to Experimental Set-Up}

``Within the context of a more realistic dynamical theory, one can use response function separations and polarization observables to enhance the sensitivity to various model dependent \emph{unobservables}, such as momentum distributions, meson exchange currents and medium modifications. One strong recent interest has been to choose kinematics in which the unobserved nucleon has a large momentum; the plane wave approximation shows that this configuration enhances sensitivity to initial-state short-range correlations (i.e. the wavefunction) and possibly quark effects. A number of these experiments have been carried out at various accelerators, but no experiments at JLab have yet reported the results."~-R. Gilman and F. Gross~\cite{Gilman:2001yh}

``The most precise constraint on these [$I=1$ exchange] currents comes from the $d\rightarrow ^1S_0$ transition, and this part of the transition is partly obscured by the poor energy resolution of the existing high $Q^2$ measurements. A new and improved experiment at JLab with higher resolution would allow the threshold $d\rightarrow ^1S_0$ process to be better extracted, with a better resulting determination of the isovector exchange currents. It is also important to determine whether or not there is a minimum near 1.2 GeV$^2$."~-R. Gilman and F. Gross~\cite{Gilman:2001yh}

``Due to the rather small size of this structure, it could have a revealing relation to certain aspects of QCD. Experiments to probe this structure and its effects in nuclei are suggested."~-J.L. Forest, et al.~\cite{Forest:1996kp}


\subsection{LiD vs He2D}

``From those results, only the $^6$Li ground state show the $LS$ coupling structure and this can be related to the $\alpha + d$ clustering in the $T=0$ state. ... The detailed analyses including the excited states are performed in our paper~\cite{Myo:2012pv}."~-T. Myo~\cite{Myo:2013dya}

\subsection{Short Range Correlations}

``... there is no clearly correct way to isolate the structure of the nucleon from the structure of the bound state. In model calculations these issues can be handled by separating the problem into two regions: at large separations ($R>R_c$) it is assumed that the system separates into two nucleons interacting through one pion exchange, and at small distances ($R<R_c$) the system is assumed to coalesce into a six-quark bag with all the quarks treated on an equal footing."~-R. Gilman and F. Gross~\cite{Gilman:2001yh}


``Calculations based on quark degrees of freedom must confront the fact that the deuteron is at least a six-quark system. Since the six quarks are identical (because of internal symmetries) the system must be antisymmetrized, and it is not clear that the nucleon should retain its identity in the presence of another nucleon."~-R. Gilman and F. Gross~\cite{Gilman:2001yh}


``SRCs are considered one of the most elusive features of the ground state nuclear wave functions."~-M. Sargsian~\cite{Sargsian:2012gj}

``One of the methods in probing 2N SRCs is studying high $Q^2$ inclusive $A(e,e')X$ scattering at $x>1.4$ in which case virtual photon scatters off the bound nucleon with momenta exceeding $k_F(A)$~\cite{Sargsian:2001ax, Sargsian:2002wc}."~-M. Sargsian~\cite{Sargsian:2012gj}

``However, it is worth noting that there is a growing activity in studies of NN bound systems at short distances by probing the short range correlations in the nuclear wave functions (see e.g.~\cite{Frankfurt:2008zv,Frankfurt:1993sp,Arrington:2011xs,Sargsian:2009hf,Boeglin:2011mt}). Currently these studies unambigously identified the tensor component of short-range $NN$ interactions in the nuclear medium. The planned experiments at hte 12 GeV energy upgraded Jefferson Lab \cite{Sargsian:2002wc,thomas2000science} will be able to probe the bound $NN$ systems at distances relevant to the nuclear core, where one may expect the onset of QCD degrees of freedom in the similar way as in the hard NN interactions.

``Overall new experiments in studies of both hard $NN$ scattering processes and $NN$ short-range correlations in nuclei will provide the necessary ground for advancing the understanding of QCD dynamics of strong forces at short distances."~-M. Sargsian~\cite{Sargsian:2014bwa}


\subsection{Notation and Conventions}
``The cross section for the double scattering process can be written as~\cite{Arnold:1979cg}
\begin{dmath}
	\frac{d\sigma}{d\Omega d\Omega_2} = \left. \frac{d\sigma}{d\Omega d\Omega_2}\right|_0 \left[1 + \frac{3}{2}hp_xA_y\sin{\phi_2} + \frac{1}{\sqrt{2}}t_{20}A_{zz} - \frac{2}{\sqrt{3}}t_{21}A_{xz}\cos{\phi_2}+\frac{1}{\sqrt{3}}t_{22}\left( A_{xx} - A_{yy} \right) \cos{2\phi_2}  \right]
\end{dmath}
where $h=\pm 1/2$ is the polarization of the incoming electron beam, $\phi_2$ the angle between the two scattering planes (defined in the same way as the $\phi$ shown in figure 24) and $A_y$ and the $A_{ij}$ are the vector and tensor analysing powers of the second scattering. Although there is a $p_z$ component to the vector polarization, the term is omitted from equation (25) as there is no longitudinal vector analysing power; without spin precession, this term cannot be determined."~-R. Gilman and F. Gross~\cite{Gilman:2001yh}

\subsection{Tensor Polarization/Force and NN Interactions/SRC}
``The most direct evidence for tensor correlations in nuclei comes from measurements of the deuteron structure functions and tensor polarization by elastic electron scattering~\cite{Gilman:2001yh}. In essence, these measurements have mapped out the Fourier transforms of the charge densitites of the deuteron in states with spin projections $\pm1$ and 0, showing that they are very different."~-R. Schiavilla, et al.~\cite{Schiavilla:2006xx}

``The nucleon-nucleon ($NN$) interaction has strong tensor forces at long and intermediate distances caused by the pion exchange, which emerges large momentum transfer, and also strong central repulsions at short distance caused by the quark dynamics~\cite{Pieper:2001mp,Kamada:2001tv}. It is important to investigate the nuclear structure by treating these characteristics of the $NN$ interaction."~-T. Myo~\cite{Myo:2013dya}

``Another recent news from SRC studies is the observation of a strong (by factor of 20) dominance of $pn$ relative to $pp$ and $nn$ SRC's in the range of the bound nucleon momenta $k_F<p<600 \mathrm{MeV}/c$\cite{Piasetzky:2006ai,Subedi:2008zz}. This observation of was an indication that at the distances relevant to the above momentum range the NN force is dominated by tensor interaction. This gave a new meaning to the above mentioned ratios:
\begin{equation}
	A_2(A) = \frac{2\cdot \sigma_{eA}}{A\cdot \sigma_{ed}},
\end{equation}
which now represent (up to the SRC center of mass motion effect) the probability of finding 2N SRCs in the nucleus A. The observed strong disbalance of $pn$ and $pp/nn$ SRCs allowed also to suggest new approximate relation for the high momentum distribution of protons and neutrons in the nucleus A\cite{McGauley:2011qc}:
\begin{equation}
	n^A_{p/n}(p) = \frac{1}{2x_{p/n}}a_d(A,y)\cdot n_d(p)
\end{equation}
where $x_{p/n}=\frac{Z}{A}/\frac{A-Z}{A}$ and $y=\left| 1-2x_p\right|$."~-M. Sargsian~\cite{Sargsian:2012gj}


``By decreasing the separation of nucleons to $1.2-1.5$~fm one will observe the onset of the strong contribution due to two-pion exchange forces resulting in the tensor interaction. The observed magnitude of the tensor as well as strong spin-orbit interactions, however, requires an addition of the vector component to the exchanged forces whose contribution gradually increases with the decrease of NN separations and dominates the overall interaction in the region of the repulsive core at $\leq 0.5$~fm."~-M. Sargsian~\cite{Sargsian:2014bwa}


\subsection{6-quark systems, hidden color}

``The third aspect which is worth to emphasize in relation to studies of NN interaction at short distances is the dynamics of the NN-bound system probed at short distances. The deuteron studies opened up a new realm in studies of QCD dynamics of strong forces. It was realized in Refs.~\cite{Harvey:1980rva, Obukhovsky:1982ci, Brodsky:1985gu, Ji:1985ky,Kusainov:1991vn}, that the fact, that the deuteron is a colorless 6-valence-quark system, creates an additional possibility for existence of color-octet three-quark (3q) states that combine into color-singlet 6-quark combinations (referred as hidden color states). The calculations indicate that there is a substantial hidden color component in the NN bound system at short distances in which the 6q system becomes a relevant degree of freedom.

``The notion of the hidden-color component gave a new possible meaning to the NN repulsion which can be in part due to orthogonality between the initial two color octet and final two color singlet nucleons. The other implication of the hidden color component is the prediction of the large contribution from the $\Delta-\Delta$ component in the NN interaction following from the decomposition of the color-singlet 6-quark system."~-M. Sargsian~\cite{Sargsian:2014bwa}

``The hidden color component is one of the unique QCD effects in NN interaction that can not be imitated by any baryons degrees of freedom. The hypothesis of the color-neutrality of the observed strongly interacting composite systems introduces the possibilities for a new reality in which baryonic systems with high degree of compositeness (one example is NN system) contain explicitly colored three-quark clusters that combine into a color neutral object. One of the best examples is the contribution of two colored baryons, $N_c$, into the colorless NN system \cite{Harvey:1980rva,Brodsky:1985gu,Ji:1985ky}. While existence of such hidden-color components are accepted within QCD there is no clear experimental evidence yet for such components."~-M. Sargsian~\cite{Sargsian:2014bwa}

\subsection{Color transparency}
``One of the most remarkable predictions of QCD is the existence of color transparency phenomena in hard processes taking place in the nuclear environment. ... The realization that these small-sized configurations are not eigenstates of QCD Hampiltonian of free hadrons and once produced at finite energies they will evolve to normal size hadrons, suggested that the experimental verification of color transparency phenomena is more complex than first expected (see e.g. Ref. \cite{Farrar:1988me,Kopeliovich:2007mm}.

``While the Color Transparency phenomenon or the reduction of the absorption in the nuclear medium is observed for the hard production of $q\bar{q}$ systems \cite{Frankfurt:1993it,ElFassi:2012nr,Clasie:2007aa}, the similar effect is still elusive for a $qqq$ system."~-M. Sargsian~\cite{Sargsian:2014bwa}

\subsection{Applications to Astrophysics}
``Recent observations of large ($\approx 2M_{Sun}$) neutron star masses \cite{Demorest:2010bx} indicates existence of rather unreasonably stiff equation of state of the nuclear matter, which is related to the persistence of the nucleonic degrees of freedom \cite{Heiselberg:2000dn} at such high densities in which one expects plenty of inelastic transitions and strong overlap of nucleon wave functions. Such persistence is also observed in probing short-range proton-neutron correlations in the nuclei \cite{Subedi:2008zz,Piasetzky:2006ai} in which the theoretical analysis \cite{Frankfurt:2008zv} shows that for up to $\leq 1$~fm separations, the NN system has no apparent non-nucleonic component, consisting almost entirely from proton and neutron.

``It is interesting that this observation also has its reflection in the modification of partonic distributions of bound nucleons in the nuclear medium (EMC effect). Here, the recent analysis \cite{Frankfurt:2012qs} indicates a rather small modification of nucleons in the medium of heavy nuclei, which seems puzzling.

``Such a persistence of the nucleons in the high density nuclear environment can be due to the short range repulsion, since the attractive interaction will make the composite system very responsive to medium modifications."~-M. Sargsian~\cite{Sargsian:2014bwa}