\subsection{Experimental Method} %Measurement of $A_{zz}$ }

As in the case for E12-13-011, the measured double differential cross section for a spin-1 target characterized by a vector polarization $P_{z}$ and tensor polarization
$P_{zz}$ is expressed as,
\begin{equation}
\frac{d^2\sigma_p}{d\Omega dE'}=\frac{d^2\sigma_u}{d\Omega dE'}\left(1-P_zP_BA_1+\frac{1}{2}P_{zz}A_{zz}\right),
\label{eq:one}
\end{equation}
where, $\sigma_p$ ($\sigma_u$) is the polarized (unpolarized) cross section, $P_B$ is the incident electron beam polarization, and $A_1$ ($A_{zz}$) is the
vector (tensor) asymmetry of the virtual-photon deuteron cross section.  This allows us to write
the polarized tensor asymmetry with $0<P_{zz}\leq 1$ using an unpolarized electron beam as
\begin{eqnarray}
\label{Azz}
A_{zz} = \frac{2}{P_{zz}}\left(\frac{\sigma_p}{\sigma_u} - 1\right),
\end{eqnarray}
where $\sigma_p$ is the polarized cross section. The tensor polarization is given by 
\begin{equation}
P_{zz}=\frac{n_+-2n_0+n_-}{n_++n_-+n_0},
\end{equation}
where $n_m$ represents the population in the $m_z=+1$,~$-1$,~or $0$ state.

Eq. \ref{Azz} reveals that the asymmetry $A_{zz}$ compares two different cross sections measured under different polarization conditions of the target: positively tensor polarized and unpolarized.  
To obtain the relative cross section measurement in the same configuration, the same target 
cup and material will be used at alternating polarization states (polarized vs. unpolarized),  
and the magnetic field providing the quantization axis will be oriented along the beamline at all times.
This field will always be held at the same value, regardless of the target material polarization state. 
This process, identical to that used for E12-13-011, ensures that the acceptance remains consistent within the stability (10$^{-4}$) of the super conducting magnet.  


Since many of the factors involved in the cross sections cancel in
the ratio, Eq. \ref{Azz} can be expressed in terms 
of the charge normalized, efficiency corrected numbers of tensor polarized ($N_p$) and unpolarized ($N_u$) counts, 
\begin{eqnarray} \label{3}
A_{zz}&=&\frac{2}{fP_{zz}}\left(\frac{N_p}{N_u}-1\right) .
\end{eqnarray}

The dilution factor $f$ corrects for the presence of unpolarized nuclei in the target and is defined by
\begin{equation}
f=\frac{N_D\sigma_D}{N_N\sigma_N+N_D\sigma_D+\Sigma N_A\sigma_A},
\end{equation}
where $N_D$ is the number of deuterium nuclei in the target and $\sigma_D$ 
is the corresponding inclusive double differential scattering cross 
section,
$N_N$ is the nitrogen number of scattered nuclei with cross section $\sigma_N$, and
$N_A$ is the numbers of other scattering nuclei of mass number $A$ with cross section $\sigma_A$.
The denominator of the dilution factor can be written in terms of the relative volume ratio of
ND$_3$ to LHe in the target cell, otherwise known as  the packing fraction $p_f$.  In our case of a cylindrical target cell oriented along the magnetic field,
the packing fraction is exactly equivalent to the percentage of the cell length filled with ND$_3$.  
%The dilution factor is discussed in further detail in Sec. \ref{dil}.

The time necessary to achieve the desired precision $\delta A$ is:
\begin{equation}
T=\frac{N_T}{R_T}=\frac{16}{P_{zz}^2f^2\delta A_{zz}^2R_T}
\end{equation} 
where $R_T$ is the total rate and $N_T = N_p + N_u$ is the total estimated 
number of counts to achieve the uncertainty $\delta A_{zz}$.  

%See Sec.~\ref{stat} for full details of the statistical uncertainty.


\subsubsection{Statistical Uncertainty}
\label{stat}
To investigate the statistical uncertainty we start with the equation for $A_{zz}$ using
measured counts for polarized data ($N_p$) and unpolarized data ($N_u$), 
\begin{equation}
A_{zz}=\frac{2}{fP_{zz}}\left(\frac{N_p}{N_u}-1\right).
\end{equation}
The statistical error with respect to counts is then
\begin{equation}
\delta A_{zz}=\frac{2}{fP_{zz}}\sqrt{\left(\frac{\delta N_p}{N_u}\right)^2+\left(\frac{N_p\delta N_u}{N_u^2}\right)^2}.
\end{equation}


\subsubsection{Systematic Uncertainty}% in $A_{zz}$ }
\begin{table}
\begin{center}
\begin{tabular}{l|c}\hline\hline
Source                         & Systematic \\
\hline
Polarimetry                  &   9.0\%   \\
Dilution/packing fraction    &   4.0\%   \\
Radiative corrections        &   1.5\%   \\
%Detector Drift               &   1.0\%   \\
%F$_1$ structure function     &   5.0\%   \\
%Computer deadtime            &  0.5\%  \\
Charge Determination           &  1.0\%  \\
Detector resolution and efficiency & 1.0\% \\
\hline
Total  &  10\%   \\
\hline
\end{tabular}
%\caption{\label{sys}Relative systematic uncertainties for $A_{zz}$.  }
\caption{\label{error1}Estimates of the scale dependent contributions to the systematic error of $A_{zz}$.}
\end{center}
\end{table}



Table \ref{error1} shows a list of the scale dependent uncertainties contributing to the systematic error in $A_{zz}$.

%The resulting estimate in the relative uncertainty of $A_{zz}$ is 9.2\%.
%\begin{table}
%\begin{center}
%\begin{tabular}{lccc}
%&source & error (\%)\\ \hline
%&Target Polarization & 8\% \\
%&Dilution/Packing fraction & 4\% \\
%&Detector Drift & 1\% \\
%&Radiative Corrections & 1.5\% \\
%&Charge Determination & 1\% \\
%&Detector resolution and efficiency & 1\% \\\hline
%   &Total & 9.2\% \\\hline
%\end{tabular}
%\end{center}
%\caption{The systematic error estimates of the $A_{zz}$ asymmetry measurement.}
%\label{error1}
%\end{table}
%
%
%The systematic uncertainty of the asymmetry $A_{zz}$ can be estimated based on known relative uncertainties and 
%the systematic effects seen in past experiments.  
%
%\subsubsection*{Target Polarization }
%The target positive tensor polarization $P_{zz}$  is calculated using the vector polarization $P_{z}$
%using Boltzmann statistics for spin temperature equilibrium,
%\begin{equation}
%P_{zz}=2-\sqrt{4-3P_z^2}.
%\end{equation}
%The uncertainty in $P_{zz}$ depends only on the uncertainty in the NMR measurement of $P_z$.
%This leads to the expression,
%\begin{equation}
%\delta P_{zz}=\frac{3P_z}{\sqrt{4-3P_z^2}}\delta P_z.
%\end{equation}
%%The polarization uncertainty for $ND_3$ have historically about 5\%.  
%%However, with
%%new techniques in polarization uncertainty minimization\cite{DUSTIN} we anticipate to be able to achieve
%%considerable reduction.  Here we use the estimate of 4\% relative uncertainty in $P_z$ for and
%%average vector polarization of 45\% leading to a relative uncertainty in $P_{zz}$ of 7.7\%.
With careful minimization, the uncertainty in $P_z$ can be held to better than
4\%, as demonstrated in the recent g2p/GEp experiment~\cite{DUSTIN}. 
This leads to a a relative uncertainty in $P_{zz}$ of 7.7\%.  Alternatively, the tensor asymmetry can be directly extracted from the NMR lineshape as discussed in Sec.~\ref{POLTARGSEC}, with similar uncertainty.  The uncertainty from the dilution factor and packing fraction of the ammonia target contributes at the 4\% level.
%
%subsubsection*{Radiative Corrections }
The systematic effect on $A_{zz}$ due to the QED radiative corrections will be quite small.  For our measurement
there will be no polarized radiative corrections at the lepton vertex, and the unpolarized corrections are known
to better than 1.5\%.

% ---------- ELLS IS THE ABOVE SENTENCE STILL TRUE IN QE????? ----------------

Charge calibration and detector efficiencies are expected to be known better to 1\%, but the impact of time-dependent drifts in these quantities must be carefully controlled.

\subsubsection*{Time dependent factors}
Eq.~\ref{3} involves the ratio of counts, which leads to cancellation of several first order systematic effects.  However, the fact that the two data sets will not be taken simultaneously leads to a sensitivity to time dependent variations which will need to be carefully monitored and suppressed.
%
To investigate the systematic differences in the time dependent components of the
integrated counts, we need to consider the effects from calibration, efficiency, acceptance,
and luminosity between the two polarization states.

Fluctuations in luminosity due to target density variation can easily be kept to a
minimum by keeping the material beads at the same temperature for both polarization
states by control of the microwave and the LHe evaporation.  The He vapor pressure reading
can give accuracy of material temperature changes at the level of $\sim$0.1\%.
Beam rastering can also be controlled to a high degree. 
%(someone who know more about beam rastering control and errors should put something here.)

The acceptance of each cup can only change as a function of time if the magnetic field
changes.  The capacity to set and reset and hold, set-ability, the target supper conducting magnet
to a desired holding field is $\delta B /B=$0.01\%.  This implies that like the cup length $l$ and the
acceptance ${\cal A}$ for each polarization states is the same.

In order to look at the effect on $A_{zz}$ due to drifts in beam current measurement
calibration and detector efficiency 
we rewrite Eq.~\ref{3} explicitly in terms of the raw measured counts $N_1$ and $N$,
\begin{eqnarray} \label{3c}
\nonumber
A_{zz}&=&\frac{2}{fP_{zz}}\left(\frac{N^c_1}{N^c}-1\right) \\
      &=&\frac{2}{fP_{zz}}\left(\frac{Q\varepsilon l \cal{A}}{Q_1\varepsilon_1 l \cal{A}}\frac{N^1}{N}-1\right)
\end{eqnarray}
where $Q$ represents the accumulated charge, and $\varepsilon$ is the detector efficiency. The target length $l$  and acceptance $\cal{A}$ are identical in both states to first order.

We can then express $Q_1$ as the change in beam current measurement calibration that occurs in
the time it takes to collect data in one polarization state before switching such that $Q_1=Q(1-dQ)$.
In this notation $dQ$ is a dimensionless ratio of changes in different polarization states.  A similar representation
is used for drifts in detector efficiency leading to,
\begin{equation}
A_{zz}=\frac{2}{fP_{zz}}\left(\frac{N_1Q(1-dQ)\varepsilon(1-d\varepsilon)}{NQ\varepsilon}-1\right).
\end{equation}
which leads to,
\begin{equation}
A_{zz}=\frac{2}{fP_{zz}}\left(\frac{N_1}{N}(1-dQ-d\varepsilon+dQd\varepsilon)-1\right).
\end{equation}

For estimates of the $dQ$ and $d\varepsilon$ we turn to previous experimental
studies.  For HRS detector drift during JLab transversity experiment E06-010, the detector response
was measured such that the normalized yield for same condition over a three month period indicated little change ($<1$\%).
These measurement where then use to show that for short time (20 minutes periods between target spin flip),
the detector drift is estimated to be less than 1\% times the ratio of the time period between target spin flip and three months.
For the present experiment we use the same estimate except for the period between target polarization states used is
$\sim$12 hours leading to an overall drift $d\varepsilon\sim0.01\%$.  A similar approach can be used to establish an estimate
for $dQ$ using studies from the data from the (g2p/GEp) experiment resulting in $d\varepsilon\sim0.01\%$.

To express $A_{zz}$ in terms of the estimated experimental drifts in efficiency and current measurement we can write,
\begin{equation}
A_{zz}=\frac{2}{fP_{zz}}\left(\frac{N_1}{N}-1\right)\pm\frac{2}{fP_{zz}}d\xi.
\end{equation}
This leads to a contribution to $A_{zz}$ on the order of $1\times10^-3$,
\begin{equation}
dA_{zz}^{drift}=\pm\frac{2}{fP_{zz}}d\xi=\pm3.7\times10^{-3}.
\end{equation}
Though a very important contribution to the error this value allows a clean measurement of $A_{zz}=0$ at $x=0.45$
without overlap with the Hermes error bar.  For this estimate we assume only two polarization state changes in a
day.  If it is possible to increase this rate then the systematic effect in $A_{zz}$ also decreases accordingly.

Naturally detector efficiency can drift for a variety of reasons, for
example including fluctuations in gas quality, HV drift or
drifts in the spectrometers magnetic field.  All of these types of variation as can be realized both
during the experiment though monitoring as well as systematic studies of the data collected.

There can be difficult to know changes in luminosity however the identical condition of the two
polarization states minimizes the relative changes in time.  There are also checks on the consistency
of the cross section data that can be use ensuring the quality of each run used in the asymmetry analysis.






















%Systematic variation in time due to detector drift was
%studied for transversity JLab experiment E06-010. For 3 months running, all detectors in
%HRS were stable to about a 1\% level.  
%%The scintillators, drift chambers, and 
%%lead-glass shower detector are stable to $\sim$2\% in 3 months, assuming
%%no significant radiation damage or detector gas loss.
%%For the measurement of $A_{zz}$ we expect no issue with radiation damage
%%being the beam current is comparatively low and in the spectrometer.
%
%
%%\subsubsection*{Charge Determination }
%%The Beam Charge Monitor at low current are estimated to have an uncertainty 
%%at the level 1-2\%.  
%%
%The BCM/BPM for low current readouts was recently re-designed for g2p.
%This was also used for QWeak, so it is available in Hall C.
%The signal to noise ratio suppression, was designed to reach
%1\% in less than 1 s.
%The Hall-A Tungsten calorimeter will be used to further reduce the charge determination systematic.
%%Integrating over a reasonable time, the charge can be measured to ^M
%%better than 1\%.
%With the calibration of Tungsten calorimeter, the BCM will have an absolute uncertainty 
%at the level of 1-2\%, and relative uncertainty (from one target polarization period 
%to next period) should be better than 1\%, depending on the linearity and the drift of the 
%BCM response.


