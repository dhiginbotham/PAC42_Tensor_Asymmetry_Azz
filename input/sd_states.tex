It was suggested for some time that to resolve the microscopic structure  of nuclei one needs to study scattering at sufficiently large momentum transfer and large relative momenta of the produced nucleons. Effectiveness of this logic was confirmed by a series of the experiments at BNL and JLab which directly observed  short-range correlations (SRC) in a series of nuclei and established similarity of the SRC in the deuteron and heavier nuclei with $pn$ correlations giving dominant contribution.  Hence, deuteron serves as a ``hydrogen atom" for the studies of the microscopic short-range structure  of the nuclei.

Thus to achieve further progress of the field it is necessary to improve our knowledge of the deuteron wave function at high momenta and especially to separate contribution to the high momentum component of the deuteron, the contribution of S- and D-waves. Note here that the dominance of the D-wave at large range of the nucleon momenta is expected in a range of the theoretical models but experimentally it was probed in a rather indirect way via measurement of $T_{20}$ for the deuteron form factor. Still the knowledge of S/D ratio for large momenta is rather poor. Indeed all wavefunctions are constrained by low energy date to reproduce S/D ratio at small momenta while the overall probability of the D-wave in the deuteron differs by a factor up to 1.5, leading to a large difference of the S/D ratio at large momenta.

The experience with the studies of the ratios of the ($e,e’$) cross sections at $x>1$ has demonstrated an early onset of the scaling of the ratios when plotted as a function of the light-cone fraction of the struck nucleon momentum.  As a result, the ratios were providing a direct measure of the ratio of the high momentum components in nuclei.  Similarly, one can expect that in the case of scattering of the polarized deuteron we expect the early scaling for the asymmetry when plotted as a function of the minimal struck nucleon momentum or the light cone fraction in the A($e,e’$) case.
It was observed at JLab that the scaling of the ratios is setting in starting at $Q^2 \sim 1 \mathrm{~GeV}^2$ so covering the range of $Q^2$ up to 2~GeV$^2$ should be sufficient to  measure the S/D ratios in an interesting momentum range.

It is worth noting here that on the top of comparing predictions for the different wave functions, one expects to be able to distinguish between non-relativistic and light cone quantum mechanic models.  The principal difference between the models is the relation between the spectator momentum and momentum in the wave function - in the nonrelativistic model they coincide, while in the light cone model the relation is non-linear starting at $k \sim 250 \mathrm{~MeV}/c$. This difference is most clearly manifested in the scattering off the polarized deuteron due to a strong dependence of the S/D ratio on the nucleon momentum.