The deuteron is the simplest nuclear system, and in many ways it is as important to understanding bound states in QCD as the hydrogen atom was to understanding bound systems in QED.  Unlike it's atomic analogue, our understanding of the deuteron remains unsatisfying both experimentally and theoretically.  
%At low energy, one pion exchange provides a good description of the nucleon-nucleon interaction, whereas at high energies, partonic degrees of freedom are relevant.  At intermediate energies, no coherent description yet exists.  
A deeper understanding of the deuteron's tensor structure will help to clarify how the gross properties of the nucleus arise from the 
underlying partons.  This provides novel information about
nuclear structure, quark angular momentum, and the polarization of the quark sea  that is not accessible in spin-1/2 targets.  


In particular, a measurement of the deuteron's tensor structure function $b_1$ is of considerable interest since it provides a clear measure of possible exotic effects in nuclei, i.e. the extent to which the nuclear ground state deviates from being a composite of nucleons only~\cite{Khan:1991qk}.  Such a measurement is further motivated by its connection with the spin-1 angular momentum sum rule~\cite{Taneja:2011sy}.

Jefferson Lab is the ideal place to investigate
tensor structure in a deuteron target
at intermediate and large $x$.  We describe such a measurement in this proposal.
