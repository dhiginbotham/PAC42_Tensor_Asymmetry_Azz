%%%%% Set text body and margins 
%%
\documentclass[11pt]{article}
\usepackage[dvips]{graphicx,color}
\usepackage{longtable}
%\setlength{\textwidth}{16.5cm}
%\setlength{\textheight}{22.2cm}
%\setlength{\hoffset}{-.25in}
%\setlength{\voffset}{-.9in}

\begin{document}

%%%%% The following lines create the E01-012 Technical Note Title Page
%%
\thispagestyle{empty}
\renewcommand{\thefootnote}{\fnsymbol{footnote}}

%%%%% Substitute your Note number, month and year in the following:
%%
\begin{flushright}
{\small
Jefferson Lab\\
Personal note 06\\
June 2011\\}
\end{flushright}

\vspace{.8cm}

%%%%% Title and Author Information:
%%
\begin{center}
{\bf\large   
Error calculations of $A_{zz}$ and $b_1^d$}
\vspace{1cm}


P. Solvignon\\
Jefferson Lab,
Newport News, VA  23606\\

\medskip

E-mail\\
solvigno@jlab.org\\
\end{center}

%\vfill

%\begin{center}
%{\bf\large   
%Abstract }
%\end{center}

%\begin{quote}
%Summary of target mass corrections methods.
%\end{quote}

%\vfill

%%%%%%%%%%%%%%%
%% Choose"Presented at," "Contributed to" for conference papers
%% or "Submitted to" for journal papers
%%%%%%%%%%%%%%%
%\begin{center} 
%{\it (Invited talk presented at)} 
%   OR 
%{\it (Contributed to)} 
%(Text varies)
%{\it Conference Name} (spell out completely) \\
%{\it City, State, Country} (Location of Conference)\\
%{\it Month Day--Month Day, Year} 
%   (Indicate duration of conference.)\\

%OR\\

%{\it Submitted to A Journal} (Spell out name of journal.)
%\end{center}

\newpage
%%
%%%%% End of title page


%%%%% Following are the commands to create the rest of the Note.
%%
%%%%% The next two lines change the line spacing to doublespace,
%%      if you should need to do that.
%%
%\renewcommand{\baselinestretch}{2}
%\normalsize

%%%%% Your paper starts here:
%%

%% To get page numbers in the rest of the paper:
%
\pagestyle{plain}


From section 6 of Hoodbhoy, Jaffe and Manihar (Nuc. Phys. B312, p571-588, 1989), we have:


\begin{eqnarray}
\frac{d\sigma_{\parallel}^H}{dxdy} & = & K \Bigg[x F_1(x) + \Big(\frac{2}{3} - H^2\Big) x b_1(x) \Bigg]\\
\frac{d\sigma_{\perp}^H}{dxdy}    & = & K \Bigg[x F_1(x) - \Big(\frac{1}{3} - \frac{1}{2}H^2\Big) x b_1(x) \Bigg]
\label{xs} 
\end{eqnarray}

with $K = \frac{e^4 M E}{2 \pi Q^4} [1+(1-y)^2]$.

For simplicity, I will use $\sigma_{\parallel}$ for $\frac{d\sigma_{\parallel}^H}{dxdy}$ and $\sigma_{\perp}$ for $\frac{d\sigma_{\perp}^H}{dxdy}$.


From Jaffe's email, we now know that $H^2 = (P+2)/3$, where $P$ is the polarization of the target. This is where is my doubt. From the paper and Jaffe's email, they define $H$ as the polarization along the beam. So my understanding is that $P = P_z$ and it is the vector polarization. 

The tensor polarization can be extracted from $P_z$ as follows:
\begin{eqnarray}
P_{zz} = 2 - \sqrt{4 - 3 P_z^2}
\label{none} 
\end{eqnarray}

The tensor asymmetry $A_{zz}$ depends on $b_1$ and $F_1$:
\begin{eqnarray}
\frac{b_1}{F_1} = - \frac{3}{2} A_{zz}
\label{none} 
\end{eqnarray}


Working with the equations of $\sigma_{\parallel}$ and $\sigma_{\perp}$, we can isolate $A_{zz}$:
\begin{eqnarray}
\frac{\sigma_{\parallel} - \sigma_{\perp}}{\sigma_{\parallel} + 2 \sigma_{\perp}} = \frac{1}{4} P_z A_{zz}
\label{none} 
\end{eqnarray}

\vspace{0.5cm}
\underline{Calculation of $A_{zz}$ statistical error}
\begin{eqnarray}
(\delta A_{zz})^2 = \Bigg( \frac{\delta A_{zz}}{\delta \sigma_{\parallel}} \Bigg)^2 (\delta \sigma_{\parallel})^2 + \Bigg( \frac{\delta A_{zz}}{\delta \sigma_{\perp}} \Bigg)^2 (\delta \sigma_{\perp})^2
\label{none} 
\end{eqnarray}

\begin{eqnarray}
\frac{\delta A_{zz}}{\delta \sigma_{\parallel}} & = &\frac{4}{P_z} \Bigg[\frac{- (\sigma_{\parallel} - \sigma_{\perp})}{(\sigma_{\parallel} + 2 \sigma_{\perp})^2} + \frac{1}{\sigma_{\parallel} + 2 \sigma_{\perp}} \Bigg] \\
         & = & \frac{4}{P_z} \frac{3 \sigma_{\perp}}{(\sigma_{\parallel} + 2 \sigma_{\perp})^2}
\label{none} 
\end{eqnarray}


\begin{eqnarray}
\frac{\delta A_{zz}}{\delta \sigma_{\perp}} & = &\frac{4}{P_z} \Bigg[\frac{- 2 (\sigma_{\parallel} - \sigma_{\perp})}{(\sigma_{\parallel} + 2 \sigma_{\perp})^2} - \frac{1}{\sigma_{\parallel} + 2 \sigma_{\perp}} \Bigg] \\
         & = & \frac{4}{P_z} \frac{-3 \sigma_{\parallel}}{(\sigma_{\parallel} + 2 \sigma_{\perp})^2}
\label{none} 
\end{eqnarray}

\begin{eqnarray}
(\delta A_{zz})^2 = \Bigg(\frac{4}{P_z}\Bigg)^2 \Bigg[ \frac{9 \sigma_{\perp}^2 \delta \sigma_{\parallel}^2 + 9\sigma_{\parallel}^2 \delta \sigma_{\perp}^2 }{(\sigma_{\parallel} + 2 \sigma_{\perp})^4} \Bigg]
\label{none} 
\end{eqnarray}

The parallel and perpendicular cross sections have the same kinematical weight $K$. Since $b_1$ is very small compared to $F_1$ (or $A_{zz}$ is very small), we can make the assumption $\sigma_{\parallel} \approx \sigma_{\perp} \equiv \sigma$ and  $\delta \sigma_{\parallel} \approx  \delta \sigma_{\perp} \equiv \delta \sigma$. 

\begin{eqnarray}
(\delta A_{zz})^2 = \frac{9 \times 16}{P_z^2} \frac{2 \sigma^2 \delta \sigma^2}{(3 \sigma)^4} = \frac{32}{9 P_z^2} \frac{\delta \sigma^2}{\sigma^2}
\label{none} 
\end{eqnarray}

\begin{eqnarray}
\delta A_{zz} = \frac{4 \sqrt{2}}{3 P_z} \frac{\delta \sigma}{\sigma} =  \frac{4 \sqrt{2}}{3 P_z} \frac{1}{\sqrt{N}}
\label{none} 
\end{eqnarray}

I get $N$ from the unpolarized cross sections. Then I calculate the rates and the time.

\begin{eqnarray}
N = \frac{32}{9} \frac{1}{P_z^2 (\delta A_{zz}^{meas})^2}
\label{none} 
\end{eqnarray}

To get the rates as a function of the theoretical tensor asymmetry, we need to apply the dilution factors:
\begin{eqnarray}
A_{zz}^{meas} = f P_{zz} A_{zz}
\label{none} 
\end{eqnarray}

\begin{eqnarray}
N = \frac{32}{9} \frac{1}{P_z^2 (f P_{zz} \delta A_{zz})^2}
\label{none} 
\end{eqnarray}

We need $N/2$ events in parallel and perpendicular kinematics. If I had a pure deuterium target, the time need will be:
\begin{eqnarray}
T = \frac{N}{R_D} = \frac{32}{9} \frac{1}{R_D P_z^2 (f P_{zz} \delta A_{zz})^2}
\label{none} 
\end{eqnarray}

Now the deuterium rates are estimated from the unpolarized deuteron cross section $\sigma_D$:
\begin{eqnarray}
 R_D = \sigma_D~dp~d\Omega~L = \sigma_D~dp~d\Omega~\rho_D~\frac{I}{e}
\label{none} 
\end{eqnarray}

with $\rho_D = \rho_{LiD} \cdot f_{LiD} \cdot PF_{LiD}$, where $f_{LiD}$ is the dilution and $PF_{LiD}$ is the packing fraction.


%%%%% Acknowledgments

%\section*{Acknowledgments}

%

%%%%% Bibliography
%%
%\begin{thebibliography}{6}

%\bibitem{OPE} H. Georgi and H. D. Politzer, Phys. Rev. {\bf D14}, 1829 (1976)

%\bibitem{KULAGIN} S. A. Kulagin and R. Petti, hep-ph 0412425

%\bibitem{BITAR} K. Bitar, P. W. Johnson and W.-K. Tung, Phys. Lett. {\bf B83}, 114 (1979)

%\bibitem{WALLY} W. Melnitchouk and F. M. Steffens (work in progress)

%\bibitem{ELLIS} R. K. Ellis, W. Furmanski and R. Petronzio, Nucl.Phys. {\bf B212}, 29 (1983)

%\bibitem{QIU} A. Accardi and J.-W. Qiu (work in progress)

%\end{thebibliography} 
%%
%%%%% End Bibliography

\end{document}
%%
%%%%% End tech_note.tex
%%%%% EOF



